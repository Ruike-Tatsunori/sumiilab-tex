\documentclass{sumiilab-paper}
%% uplatex を使う場合:
% \documentclass[uplatex]{sumiilab-paper}
\usepackage{amsmath,amssymb,amsfonts,ascmac,cases,bm}
\usepackage{cite}
\usepackage{enumerate}

%% ===============================================
%% 論文の表紙に表示される情報
%% ===============================================

% 論文の年度と種類
\paper{平成 n 年度 卒業論文}% 学部生
%\paper{平成 n 年度 修士論文}% 修士

% 論文のタイトル
\title{住井研究室の\\ステキな論文クラスファイルの使用例}

% 学籍番号と著者のお名前
\author{X0XX1234 ラムダ 小太郎}

% 著者の所属
\institute{東北大学 工学部\\情報知能システム総合学科}% 学部生
%\institute{東北大学 大学院 情報科学研究科\\情報基礎科学専攻}% 修士

% 指導教員のお名前
\supervisor{住井 英二郎 教授}% 指導教員
%\subsupervisor{xxx xxx 准教授}% 論文指導教員(省略可)

% 論文発表日時
\date{平成 n 年1月1日 \quad 23:00--23:30}
% 発表場所
\venue{電子情報システム・応物系1号館2階トイレ}

%% ===============================================
%% ソースコードの設定
%% ===============================================

% プログラミング言語と表示するフォント等の設定
\lstset{
  language={[Objective]Caml},% プログラミング言語
  basicstyle={\ttfamily\small},% ソースコードのテキストのスタイル
  keywordstyle={\bfseries},% 予約語等のキーワードのスタイル
  commentstyle={},% コメントのスタイル
  stringstyle={},% 文字列のスタイル
  frame=trlb,% ソースコードの枠線の設定 (none だと非表示)
  numbers=left,% 行番号の表示 (none だと非表示)
  numberstyle={\footnotesize},% 行番号のスタイル
  xleftmargin=15pt,% 左余白
  xrightmargin=5pt,% 右余白
  keepspaces=true,% 空白を維持する
  mathescape,% $ で囲った部分を数式として表示する ($ がソースコード中で使えなくなるので注意)
  % 手動強調表示の設定
  moredelim=[is][\bfseries]{@*}{*@},
  moredelim=[is][\itshape]{@/}{/@}
}

\begin{document}
\maketitle

\begin{abstract}
ステキな論文の概要
\end{abstract}

\chapter*{謝辞}

ステキな論文の謝辞

%% 目次
\tableofcontents

%% ここから本文

\chapter{序論}

%% 参考文献は \cite{ID} とします(ID は refs.bib 内で文献につけた識別子)
%% BibTeX の使い方などは各自調べて下さい。
序論とか本論とか結論とか \cite{Pierce:TypeSystems}

\chapter{本論}

\section{ソースコード}

ソースコード\ref{src:listup_nodes}は二分木を深さ優先探索して、ノードを列挙する関数である。
\begin{lstlisting}[caption=二分木のノードのリストアップ,label=src:listup_nodes]
type 'a bin_tree =
  | Leaf of 'a
  | Node of 'a bin_tree * 'a bin_tree

let rec listup_nodes = function
  | Leaf x -> [x]
  | Node (r, l) -> (listup_nodes r) @ (listup_nodes l)
\end{lstlisting}
ソースコードの書き方等については slide ブランチの slide.tex を参照されたし。

\section{定理環境}

\begin{theorem}[定理のタイトル]
  定理の内容
\end{theorem}

\begin{lemma}[補題のタイトル]
  補題の内容
\end{lemma}

\begin{corollary}[系のタイトル]
  系の内容
\end{corollary}

\begin{proposition}[命題のタイトル]
  命題の内容
\end{proposition}

\begin{definition}[定義のタイトル]
  定義の内容
\end{definition}

\begin{example}[例のタイトル]
  例の内容
\end{example}

\begin{assumption}[仮定のタイトル]
  仮定の内容
\end{assumption}

\begin{axiom}[公理のタイトル]
  公理の内容
\end{axiom}

\begin{proof}[証明のタイトル]
  証明の内容 \qed
\end{proof}

\chapter{結論}

%% 参考文献: bibtex
\bibliographystyle{junsrt}
\bibliography{refs}

\end{document}
